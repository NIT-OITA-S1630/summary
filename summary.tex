%\documentclass{nitocs}
\documentclass[summary]{nitocs}

\usepackage[dvipdfmx]{graphicx}
\usepackage{latexsym}
% 
\def\Underline{\setbox0\hbox\bgroup\let\\\endUnderline}
\def\endUnderline{\vphantom{y}\egroup\smash{\underline{\box0}}\\}
\def\|{\verb|}



\setcounter{page}{1}
\begin{document}

    \title{タイトル}
    \author{橋本 燎}{Ryo Hashimoto} % 氏名
    \advisor{井上 優良}{Yusuke Inoue}   % 指導教員氏名
    %! 実行時の日付が自動挿入されるはず
    \date{令和3年\number\month 月\number\day 日} 

    \begin{keyword}
    キーワード
    \end{keyword}


    \maketitle
    \section{卒業研究抄録のフォーマット指針}
        \label{sec:format}
        以下,卒業研究抄録用スタイルファイルを用いた抄録フォーマットの指針について述べるので,
        これに従って原稿を用意頂きたい.\LaTeX を用いた一般的な文章作成技術については,\cite{okumura, companion} 等を参考にされたい.

    %4
    \section{抄録の構成}
        \label{config}

        抄録の\LaTeX ソースコードの基本的な構成が次のようになる.

        \noindent
        \|\documentclass[summary]{nitocs}|.
        \|\begin{document}|\\
        \|\title{表題(和文)}|\\
        \|\author{情報 太郎}{Joho Taro}|\\
        \|\advisor{処理 花子}{Shori Hanako}|\\
        \|\date{平成27年12月1日}|\\
        \|\begin{jkeyword}|\\
        \|<キーワード1>,<キーワド2>,…|\\
        \|\end{jkeyword}|\\
        \|\maketitle|\\
        \|\section{|第1節の表題\|}|\\
        \dots\dots\dots\dots\dots\\
        \quad \|<本文>|\\
        \dots\dots\dots\dots\dots\\
        \|\begin{thebibliography}{99}%9 or 99|\\
        \|\bibitem{1}|\\
        \|\bibitem{2}|\\
        \|\end{thebibliography}|\\\\
        \noindent
        \|\end{document}|



    \section{本文}
        \subsection{見出し}

            節や小節の見出しには \|\section|, \|\subsection|, \|\subsubsection|,
            \|\paragraph| といったコマンドを使用する.

        \subsection{行送り}

            2段組を採用しており,左右の段で行の基準線の位置が一致することを原則とし
            ている.また,節見出しなど,行の間隔を他よりたくさんとった方が読みやすい
            場所では,この原則を守るようにスタイルファイルが自動的にスペースを挿入す
            る.したがって本文中では \|\vspace| や \|\vskip| を用いたスペースの調整
            を行なわないようにすること.


        \subsection{句読点}

            句点には全角の「.」,読点には全角の「,」を用いる.ただし英文中や数式中
            で「.」や「,」を使う場合には,半角文字を使う.「。」や「、」は使わない.

    %4.3.5
        \subsection{全角文字と半角文字}

            全角文字と半角文字の両方にある文字は次のように使い分ける.

            \begin{enumerate}
                \item 括弧は全角の「(」と「)」を用いる.但し,英文の概要,図表見出し,
                書誌データでは半角の「(」と「)」を用いる.

                \item 英数字,空白,記号類は半角文字を用いる.ただし,句読点に関しては,
                前項で述べたような例外がある.

                \item カタカナは全角文字を用いる.

                \item 引用符では開きと閉じを区別する.
                開きには \|``| を用い,閉じには\|''| を用いる.
            \end{enumerate}


        \subsection{箇条書}

            箇条書に関する形式を特に定めていない.場合に応じて標準的な \|enumerate|,
            \|itemize|, \|description| の環境を用いてよい.

    \section{数式}\label{sec:Item}


        \subsection{本文中の数式}

            本文中の数式は \|$| と \|$|, \|\(| と \|\)|, あるいは \|math| 環境のいず
            れで囲んでもよい.


        \subsection{別組の数式}

            別組数式(displayed math)については \|$$| と \|$$| は使用せずに,\|\[| と 
            \|\]| で囲むか,\|displaymath|, \|equation|, \|eqnarray| のいずれかの環
            境を用いる.これらは
            %
            \begin{equation}
            \Delta_l = \sum_{i=l|1}^L\delta_{pi}
            \end{equation}
            %
            のように,センタリングではなく固定字下げで数式を出力し,かつ背が高い数式
            による行送りの乱れを吸収する機能がある.


        \subsection{eqnarray環境}

            互いに関連する別組の数式が2行以上連続して現れる場合には,単に\|\[| と 
            \|\]|,あるいは \|\begin{equation}| と\|\end{equation}| で囲った数式を書
            き並べるのではなく,\|\begin|\allowbreak\|{eqnarray}| と 
            \|\end{eqnarray}| を使って,等号(あるいは不等号)の位置で縦揃えを行なっ
            た方が読みやすい.
            \begin{figure}[tb]
                \setbox0\vbox{
                \hbox{\|\begin{figure}[tb]|}
                \hbox{\quad \|<|図本体の指定\|>|}
                \hbox{\|\caption{<|和文見出し\|>}|}
                %\hbox{\|\ecaption{<|英文見出し\|>}|}
                \hbox{\|\label{| $\ldots$ \|}|}
                \hbox{\|\end{figure}|}
                }
                \centerline{\fbox{\box0}}
                \caption{1段幅の図}
                %\ecaption{Single column figure with caption\\
                %explicitly broken by $\backslash\backslash$.}
                \label{fig:single}
            \end{figure}

    \section{図}

        1段の幅におさまる図は,\figref{fig:single} の形式で指定する.位置の指定
        に \|h| は使わない.また,図の下に和文の見出しを,
        \|\caption| で指定する.文字数が多い見出しはは自動的に改
        行して最大幅の行を基準にセンタリングするが,見出しが2行になる場合には適
        宜 \|\\| を挿入して改行したほうが良い結果となることがしばしばある
        (\figref{fig:single} の英文見出しを参照).図の参照は \|\figref{<|ラベ
        ル\|>}| を用いて行なう.


        図の中身では本文と違い,どのような大きさのフォントを使用しても構わない.
        また図の中身として,encapsulate された
        PostScriptファイル(いわゆるEPSファイル)を読み込むこともできる.読み込
        みのためには,プリアンブルで
        %
        \begin{quote}
            \|\usepackage{graphicx}|
        \end{quote}
        %
        を行った上で,\|\includegraphics| コマンドを図を埋め込む箇所に置き,その
        引数にファイル名(など)を指定する.

    %4.6
    \section{表}

        表の罫線はなるべく少なくするのが,仕上がりをすっきりさせるコツである.罫
        線をつける場合には,一番上の罫線には二重線を使い,左右の端には縦の罫線を
        つけない (\tabref{tab:example}).表中のフォントサイズのデフォルトは
        \|\footnotesize|である.

        また,表の上に和文の見出しを, \|\caption| で
        指定する.表の参照は \|\tabref{<|ラベル\|>}| を用いて行なう.

        \begin{table}[tb] 
            \caption{表の例} 
            %\ecaption{An Example of Table.}
            \label{tab:example}
            \hbox to\hsize{\hfil
            \begin{tabular}{l|lll}\hline\hline
            & column1 & column2 & column3 \\\hline
            row1 &	item 1,1 & item 2,1 & ---\\
            row2 &	---      & item 2,2 & item 3,2 \\
            row3 &	item 1,3 & item 2,3 & item 3,3 \\
            row4 &	item 1,4 & item 2,4 & item 3,4 \\\hline
            \end{tabular}\hfil}
        \end{table}


    \section{参考文献}

        本文中で参考文献を参照する場合には%,%参考文献番号が文中の単語として使われ
        %る場合と,そうでない参照とでは,使用する文字の大きさが異なる.前者は
        %\|\Cite|により参照し,後者は
        %\|\cite|により参照する.たとえば;
        \|\cite|を使用する.参照されたラベルは自動的にソートされ,
        \|[]|でそれぞれ区切られる.
        %
        \begin{quote}
            文献 \|\cite{companion,okumura}| は \LaTeX の総合的な解説書である.
        \end{quote}
        %
        と書くと;
        %
        \begin{quote}
            文献\cite{companion,okumura}は \LaTeX の総合的な解説書である.
        \end{quote}
        %
        が得られる.




    \begin{thebibliography}{10}

    \bibitem{latex}
    Lamport, L.: {\em A Document Preparation System \LaTeX User's Guide \&
    Reference Manual}, Addison Wesley, Reading, Massachusetts (1986).
    (Cooke, E., et al.訳:文書処理システム \LaTeX,アスキー出版局
    (1990)).

    \bibitem{total}
    伊藤和人: \LaTeX トータルガイド,秀和システムトレーディング (1991).
    \bibitem{nodera}
    野寺隆志:楽々 \LaTeX,共立出版 (1990).

    \bibitem{okumura}
    奥村晴彦:改訂第5版 \LaTeXe 美文書作成入門,
    技術評論社(2010).

    \bibitem{companion}
    Goossens, M., Mittelbach, F. and Samarin, A.:
    {\it The LaTeX Companion},
    Addison Wesley, Reading, Massachusetts (1993).

    \bibitem{book1}
    木下是雄:
    理科系の作文技術,
    中公新書(1981).

    \bibitem{book2}
    Strunk W. J. and White E.B.:
    {\it The Elements of Style, Forth Edition},
    Longman (2000).

    \bibitem{book3}
    Blake G. and Bly R.W.:
    {\it The Elements of Technical Writing},
    Longman (1993).

    \bibitem{book4}
    Higham N.J.:
    {\it Handbook of Writing for the Mathematical Sciences},
    SIAM (1998).

    \end{thebibliography}

\end{document}
